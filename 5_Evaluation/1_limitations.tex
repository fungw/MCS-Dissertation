\section{Limitations}
The main limitation to the evaluation of the simulation is the computing power required to complete a simulation run. Numerous computers were used over the course of this dissertation. In this section, a timeline of the process of running the simulations and their limitations is explained.

Initially, a personal MacBook Pro was used. VMWare allowed the installation of Ubuntu 16.04. \ac{SUMO}, \ac{VEINS} and \ac{OMNeT++} was installed on the virtual machine. However, the MacBook Pro began exhibiting random shut downs. The random shut downs persisted, attempts were made to try and come to a resolution to the problem, but to no avail.

University computers in the computer science labs were then used for the simulations. However, estimations made for the duration of one complete run of simulations was assumed to be much lower. An attempt was made to scale down all the data, this included the parking spaces available, number of vehicles inserted into the simulation as well as the parking duration of vehicles. Attempts were also made to run the simulation without the simulators' \ac{GUI} and through the \ac{CLI}. However, an estimation of the simulation duration to complete exceeded three weeks.

2 University VMs were then acquired to run the simulations. The VMs had to be set up and Ubuntu, along with \ac{SUMO}, \ac{VEINS} and \ac{OMNeT++} were installed on them. All simulations were run from the \ac{CLI}. A list of the specifications of each machine used is outlined in section \ref{sec:power}.