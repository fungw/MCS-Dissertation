\section{Background}
Kuhn defines a paradigm shift as a fundamental change in the basic concepts and experimental practices of scientific disciplines \citep{Kuhn1970TheRevolutions}. In this paper \citep{Eltoweissy2010TowardsPaper}, it mentions that within the computing industry, various infrastructure providers with large computing resources are often underutilised. The paper highlights the shift in which business models and computing resource allocations have shifted towards an on-demand model. This shift is known today as cloud computing. The same paper also envisions a shift in the ever increasing fleet of vehicles on the roads. In that, they could potentially shift towards a system that can support a networking environment.

\begin{displayquote}
    \textit{The huge fleet of energy-sufficient vehicles that crisscross our roadways, airways, and waterways, most of them with a permanent Internet presence, featuring substantial on-board computational, storage, and sensing capabilities can be thought of as a huge farm of computers on the move.}
\end{displayquote}

The paper stresses that the computational power and sensing capabilities of vehicles, could potentially be pooled together, to \textit{autonomously self-organise into the cloud} and form what the paper has called ``Autonomous Vehicular Cloud (AVC)". The comparison is made between the previously underutilised computing resources of infrastructure providers to that of current autonomous vehicles with no vehicular networking capabilities. In a similar way where cloud computing has introduced the concept of on-demand computing resource models for consumers to utilise its resources entirely, vehicles may act in a similar way to fully utilise their real-time on-the-road sensing capabilities as well as coordination of communications and physical resources in an ad-hoc manner in traffic.