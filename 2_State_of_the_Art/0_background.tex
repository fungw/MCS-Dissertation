\section{Background}
Kuhn defines a paradigm shift as a fundamental change in the basic concepts and experimental practises of scientific disciplines \citep{Kuhn1970TheRevolutions}. In this paper \citep{Eltoweissy2010TowardsPaper}, it discusses how various infrastructure providers with large computing resources are often under-utilised. The paper highlights the shift in which business models and computing resource allocations have shifted towards an on-demand model. This shift is known today as cloud computing. Cloud computing generally falls under three categories. Software as a Service (SaaS), where end-users pay for pre-built software hosted on the vendors' infrastructure. Platform as a Service (PaaS), where end-users pay for infrastructure and programming tools in order to build their own applications. And Infrastructure as a Service (IaaS), where end-users pay for computing resources, storage and networking to host their applications or services. Additionally, the same paper envisions the ever increasing fleet of vehicles on the roads could potentially support a networking environment.

\begin{displayquote}
    \textit{The huge fleet of energy-sufficient vehicles that crisscross our roadways, airways, and waterways, most of them with a permanent Internet presence, featuring substantial on-board computational, storage, and sensing capabilities can be thought of as a huge farm of computers on the move.}
\end{displayquote}

The paper stresses that the computational power and sensing capabilities of autonomous vehicles today, could potentially be pooled together, to \textit{autonomously self-organise into the cloud} and form what the paper has called ``Autonomous Vehicular Cloud (AVC)". The comparison is made between the previously under-utilised computing resources of infrastructure providers to that of current autonomous vehicles today with no vehicular networking capabilities. In a similar way where cloud computing has introduced the concept of on-demand computing resource models for consumers to fully utilise its resources, vehicles may act in a similar way to fully-utilise their real-time on-the-road sensing capabilities as well as coordination of communications and physical resources in an ad-hoc manner in traffic.+