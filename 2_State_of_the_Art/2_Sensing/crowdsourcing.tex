\subsubsection*{Crowdsensing}
Crowdsensing can be defined as the collection of data from citizens to provide a dataset where useful information can be extracted from \cite{Villanueva2016CrowdsensingMonitoring}. 

ParkJam \citep{Kopeck2012PARKJAM:Demo} is a proposal for a crowdsensing parking space app. It uses publicly available geographic data to get parking areas. The app relies on participating users to submit information regarding parking space availabilities. Incentives for users to participate in updating the parking spaces include receiving information regarding vacant parking space in return.

While crowdsensing can provide up-to-date parking data, it requires constant user attention and participation. ParkJam could be utilised more efficiently if users are not required to input the data manually. PhonePark\cite{xu_real-time_2013}, as mentioned in the previous accelerometer section, utilises mobile phone accelerometers to detect whether a driver has parked. This information could be relayed to a central system and broadcasted to other participating drivers. As a result, PhonePark is essentially a crowdsensing application with accelerometer parking space occupancy detector.