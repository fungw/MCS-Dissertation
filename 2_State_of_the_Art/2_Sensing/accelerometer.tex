\subsubsection*{Accelerometer}
Accelerometers are very common in modern phones. PhonePark is a solution that follows a real-time analysis of device mode transitions in order to detect parking space occupancies \citep{xu_real-time_2013}. It works by utilising information of a users' mobile phone. PhonePark proposes three detection methods for parking space occupancies. These are listed below with an overview of how they work.

\begin{enumerate}
    \item \textit{Bluetooth}: This method of detection involves the mobile phone owner to connect to a vehicles Bluetooth system. If the device is tethered to the vehicles, then it assumes that the vehicle is being driven. When the Bluetooth disconnects, as the driver walks (10 meters) away from the vehicle, it infers that the car is parked. 
    \item \textit{Transition Models}: Different states are used to classify whether a user is driving, walking or stationary. If the transition sequence of driving to stationary to walking is observed. PhonePark infers that the vehicle is recently parked. The phones accelerometer is used to make estimations as to whether the user is driving, walking or stationary.
    \item \textit{Pay-by-Phone Piggyback}: The method for parking detection with pay-by-phone piggyback is to allow the user to pay through their mobile phone. Upon payment, the user will be asked for their parking space number. This is forwarded to the pay-by-phone and the parking space occupancy is detected in this way.
\end{enumerate}
