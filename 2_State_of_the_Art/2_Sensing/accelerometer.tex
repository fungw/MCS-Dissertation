\subsubsection*{Accelerometer}
Accelerometers can be found in many smartphones. Accelerometers can be used to detect a phone's orientation, motion and rotation. Information regarding a phones' orientation can allow the phone's screen to adjust to landscape mode or portrait mode. Accelerometers in mobile phones can also be used to recognise activity of a user \citep{Brezmes2009ActivityPhone}.

PhonePark is a solution that includes a real-time analysis of device mode transitions to detect parking space occupancies \citep{xu_real-time_2013}. It works by utilising information of a users' mobile phone. PhonePark proposes three detection methods of parking space occupancies. These are listed below with an overview of how they work.

\begin{enumerate}
    \item \textit{Bluetooth}: This method of detection involves a mobile phone tethered to a vehicles' Bluetooth system. If the device is tethered to the vehicle, then it assumes that the vehicle is being driven. When the Bluetooth disconnects, as the driver walks (10 meters) away from the vehicle, it infers that the vehicle is parked. 
    \item \textit{Transition Models}: Different states are used to classify whether a user is driving, walking or stationary. If a transition sequence of driving to stationary, and stationary to walking is observed. PhonePark infers that the vehicle is recently parked. The phone's accelerometer is used to make estimations as to whether the user is driving, walking or stationary.
    \item \textit{Pay-by-Phone Piggyback}: Pay-by-phone piggyback is to allow the user to pay through their mobile phone. Upon payment, the user will be asked for their parking space number. This information is forwarded from the pay-by-phone company to a central system that keeps track of the parking spaces. Thus the parking space occupancy can be detected in this way.
\end{enumerate}

While the methods of detection are possible, the paper acknowledges that not all drivers have mobile phones. Thus the solution provided by PhonePark is not always possible. Additionally, the paper acknowledges GPS errors, Bluetooth pairing difficulties and incorrect transition classifications. All these errors contribute to inaccurately detecting parking space occupancies.