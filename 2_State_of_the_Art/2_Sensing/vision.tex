\subsubsection*{Computer Vision}
In this paper \citep{cho_automatic_2016}, it discusses taking advantage of CCTV cameras inside parking lots in order to guide vehicles to a parking spot. The system proposes a central computing unit that a communicates and guides vehicles to a vacant parking spot within the parking lot. The work was tested within a lab environment with an USB 3.0 camera. The identification of parking spaces are identified by placing a coloured piece of paper in the centre of each parking location. The coloured paper is hidden when the body of a vehicle is in its position. Thus, are able to  identify whether a parking spot is vacant or occupied. However, the paper concludes that dynamic characteristics, such as camera image noise and geometric orientations impede its use in general parking lot scenarios.

In this paper \citep{amato_deep_2017}, it proposes a deep learning decentralised parking lot occupancy detector. The solution is based on a deep \ac{CNN} designed for use on smart cameras. Smart cameras are defined as ``cameras capable of processing the acquired images and transmitting just the result to a remote server". The proposed solution is able to learn where parking spaces are and detect whether a vehicle is parked within it or not. The advantages of designing a decentralised detection system is that it may be deployed to numerous amounts of cameras and learn by itself to detect parking space occupancies. This research took place on publicly available parking lot datasets, recorded in different environments and weather conditions. This paper concludes that its solution outperforms other existing solutions within the field of parking space occupancy detection with computer vision.