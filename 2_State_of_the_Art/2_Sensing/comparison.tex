\subsubsection*{Comparison of Sensing Technologies}
Table \ref{table:stationary_sensors} and \ref{table:mobile_sensors} illustrates general considerations put forward when comparing parking sensor technologies as defined in this paper \citep{Lin2008SecurityNetworks}. Additional information is obtained from \citep{dokur_embedded_2016} regarding the cost and accuracy of sensors mentioned. \\

\noindent An explanation of the columns of tables \ref{table:stationary_sensors} and \ref{table:mobile_sensors} are as follows:
\begin{description}[leftmargin=14em, style=nextline]
    \item[Intrusive Installation (IV)] Intrusive installation is defined as whether the sensor needs to be embedded into the road. This could yield significant costs. For example, on-street ultrasonic sensors may require installers to dig up the road to install the sensors securely into the ground.
    \item[Multiple Detection (MD)] The ability to detect multiple parking space vacancies at once.
    \item[Cost (C)] The cost of each sensor is an important consideration. This also depends on whether the sensor can detect multiple spaces. Detection of multiple spaces would bypass the need for a single sensor per parking spot. Additionally, if the sensor requires intrusive installation, then the cost is likely to increase.
    \item[Accuracy (A)] The accuracy of a sensor device depends on environmental factors, as well as the method for detecting parking spaces.
\end{description}

In tables \ref{table:stationary_sensors} and \ref{table:mobile_sensors}, the discussed parking sensors are compared.

\subsubsection*{\underline{Stationary Sensors}}
\begin{table}[H]
    \begin{center}
        \begin{tabularx}{\textwidth}{|X|c|c|c|c|c|}
            \hline
            Sensor & C & IV & MD & A \\
            \hline\hline
            Accelerometer & * & \xmark & \checkmark & * \\ \hline
            Acoustic & *** & \xmark & \checkmark & * \\ \hline
            Fibre Optics & * & \xmark & \xmark & ** \\ \hline
            Magnetometer & ** & \checkmark & \xmark & *** \\ \hline
            RFID & ** & \checkmark & \xmark & ** \\ \hline
            Ultrasonic & * & \xmark & \checkmark & *** \\ \hline
            Vision & *** & \xmark & \checkmark & ** \\ \hline
        \end{tabularx}
        \caption{Stationary Sensors Comparison Table}
        \label{table:stationary_sensors}
    \end{center}
\end{table}

\subsubsection*{\underline{Mobile Sensors}}
\begin{table}[H]
    \begin{center}
        \begin{tabularx}{\textwidth}{|X|c|c|c|c|}
            \hline
            Sensor & C & IV & MD & A \\
            \hline
            Crowdsource (Smartphone) & \xmark & \xmark & \checkmark & \xmark \\ \hline
            Laser Range Sensor & *** & \xmark & \checkmark & *** \\ \hline
            Ultrasonic & * & \xmark & \checkmark & *** \\ \hline
        \end{tabularx}
        \caption{Mobile Sensors Comparison Table}
        \label{table:mobile_sensors}
    \end{center}
\end{table}

\subsubsection{Conclusion}
A wide range of parking space sensing technologies has been discussed. There are many ways in which parking space occupancies can be detected. From the unconventional ways of detection with fibre optics and laser range sensors to the utilisation of ultrasonic sensors and \ac{RFID} readers. Each and every sensing solution provide adequate results for the detection of occupied parking spaces.