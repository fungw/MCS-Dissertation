\subsubsection*{Laser Scanners}
Laser range finders are used to detect the distances from a source point into the distance. In a research paper \citep{ono_probe_2002}, a test vehicle with laser range finders attached to the side of test vehicle makes an attempt to detect parking space occupancies. The laser scanners are able to retrieve depth information from the side of the test vehicle to the side of the street. With a known location of parking spaces, it would be able to analyse the distance between the test vehicle to the side of the street at a known parking spot location. Thus, it would be able to infer whether or not the parking space is occupied by analysing the distance between the test vehicle and the side of street. If the distance exceeds the distance between the test vehicle and the curb, then its likely that the space is vacant. Whereas if the distance is less than the expected distance between the test vehicle and the curb, then it is likely taken.

At times, the detection of black vehicles go unnoticed. Thus, this paper has devised two algorithms to further enhance the analysis of a normalised depth image obtained from the laser range finder. They are known as the height-curve method and the depth-curve method. The methods include computer vision techniques, such as edge detection in order to obtain additional information from the depth images. With the depth-curve method, it is able to find the edge of the road, and if a vehicle is present, it is able to detect the outline of the vehicle. Using the height-curve method, it is able to infer whether the height of outline obtained through the depth-curve method is a vehicle or a wall. Through the use of both methods, the study concludes that it provides an accurate solution for the detection of parking space occupancies with a laser range finder.

While the use of laser-range finders are possible in the detection of parking spaces. It requires a vehicle to drive around, scanning all known parking spaces in order to provide updates. One idea would be to attach these laser range-finders to the side of taxis. Similarly, they could be attached to other forms of public transport systems. However, taxis are expected to cover more ground and circulate more granular areas within an urban environment.