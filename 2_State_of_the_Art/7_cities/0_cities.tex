\section{Real World Examples}
In order to fully encapsulate the idea of smart parking, an analysis of how it is implemented in other cities should also be explored. Listed below are some examples of already existent cities that have smart parking facilities deployed in their cities. Some of them are pilot schemes while others are outsourced by private firms to introduce such a system into their society.

\subsection*{Westminster - London (October 2014)}
SmartParking \citep{18} is a company that provides solutions for cities. It is involved in a few cities around the world. The geographically closest city that SmartParking is involved in is Westminster, London.

The deployment of 3,400 RFID intrusive sensors are installed around the city. Each parking sensor monitors its corresponding spot. Zone unit monitors installed in street lamps monitor each sector. Similar model as the Voronoi Model mentioned in previous sections. These sensors sit on a 3G infrastructure that allows communication to the central traffic controller in the Westminster district. 

The service is accessible to drivers from an app, where they can receive real-time parking information as well as pay for spaces via their mobile phones.

The point of analysing how other cities solve smart parking is to expand the knowledge in how different cities under their specific circumstances adapt to allow such a system to exist within their society. Thus, through research, experiment and simulation, along with comparing the cities with a similar size to that of Dublin, a finer representation of such a system may be investigated.

\subsection*{San Francisco (2011)}
San Francisco has been the leading example of the implementation of a smart parking facility in its city.

Since its implementation, a study in March 2014 shows that drivers cruising looking for parking spots has been down by 50\% \citep{Millard-Ball2014IsExperiment}. In another economic study on the SFPark implementation \citep{Shriver2016UnderstandingProject}, it provides positive feedback from the dissemination of parking spaces around the city. Upheld by the demand based pricing model, the distribution of spaces have provided a wider variance in the spaces available in each sector of the city and also has managed to suppress drivers from cruising around looking for a spot. This is highly managed by the fact that the system poses to retain at least 20\% of each regions parking spaces free. This is so that the probability of spaces being available as drivers drive into the area will be high.