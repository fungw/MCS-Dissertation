\pagebreak

\section{Real World Examples}
By the end of this section, this literature review will be complete. Throughout this chapter, an analysis of how parking space sensing, data dissemination and vehicle networking can be achieved. To conclude this chapter, real world examples are analysed in this section. This involves an analysis of existing solution in cities around the world. Some of these smart parking systems are pilot schemes while others are outsourced to private firms.

\subsection*{\underline{SmartParking Limited}}
SmartParking \citep{18} is a private company that provides smart parking solutions for cities. In this section, the solutions provided by SmartParking are analysed. This is followed by a list of the cities that they currently serve.

\subsubsection{Sensors}
The sensors by SmartParking consists of a magnetic and infrared sensor. The sensors are wireless and are powered by long-life batteries. The occupancy information is broadcasted to electronic display boards, to the SmartParking app to assist drivers in finding a vacant spot. However, regarding network topology, it does not explicitly say whether a nearby \ac{RSU} oversees these sensors or in what way they communicate with the smart parking system \citep{2017SensorsSmartParking}.

\subsubsection{ANPR}
Cameras are used to oversee off-street parking locations. \ac{ANPR} is used to detect the number plates of vehicles. This solution allows for barrier-free parking lots \citep{2017ANPRSmartParking}.

\subsubsection{RFID}
SmartParking also offers \ac{RFID} solutions for both on-street and off-street locations. For on-street locations, permit holders can simply attach a \ac{RFID} tag onto their vehicles to allow parking enforcers verify that they are authorised to park in that spot.

\ac{RFID} tags can also be extended to vehicles in an off-street scenario. Vehicles may require an authorised tag to park in a specific space within a parking lot complex \citep{2017RFIDSmartParking}.

\subsubsection{Westminster - London (October 2014)}
The deployment of 10,000 RFID intrusive sensors is installed around the city. Zone unit monitors are installed in street lamps monitor each sector. The service is accessible to drivers from an app, where they can receive real-time parking information as well as pay for spaces via their mobile phones \citep{2013WestminsterSmartParking}.

\subsubsection{Cardiff - Wales (January 2017)}
The deployment of 3,000 Infrared/Magnetometer sensors. The parking information is accessible through an app and from electronic display boards. \ac{ANPR} recognition technology are also deployed in off-street parking locations \citep{2017CardiffSmartParking}.

\subsection*{\underline{Other solutions}}
\subsubsection{San Francisco (2011)}
San Francisco has been the leading example of the implementation of a smart parking facility.

Since its implementation, a study in March 2014 shows that drivers cruising looking for parking spots have been down by 50\% \citep{Millard-Ball2014IsExperiment}. In another economic study on the SFPark implementation \citep{Shriver2016UnderstandingProject}, it provides positive feedback from the dissemination of parking spaces around the city. Supported by the demand-based pricing model, the distribution of spaces have provided a wider variance in the spaces available in each sector of the city and also has managed to suppress drivers from cruising around looking for a spot. This is possible because the system poses to retain at least 20\% of each regions parking spaces free. This is so that the probability of spaces being available as drivers drive into the area will be high.